% This LaTeX was auto-generated from MATLAB code.
% To make changes, update the MATLAB code and export to LaTeX again.

\documentclass{article}

\usepackage[utf8]{inputenc}
\usepackage[T1]{fontenc}
\usepackage{lmodern}
\usepackage{graphicx}
\usepackage{color}
\usepackage{listings}
\usepackage{hyperref}
\usepackage{amsmath}
\usepackage{amsfonts}
\usepackage{epstopdf}
\usepackage[table]{xcolor}
\usepackage{matlab}

\sloppy
\epstopdfsetup{outdir=./}
\graphicspath{ {./SIO175_Finalcode_images/} }

\begin{document}

\begin{par}
\begin{flushleft}
\texttt{\textbf{Tyler Barbero}}
\end{flushleft}
\end{par}

\begin{par}
\begin{flushleft}
\texttt{\textbf{SIO175 Final Project (code)}}
\end{flushleft}
\end{par}

\begin{par}
\begin{flushleft}
\texttt{\textbf{Topic: Jet Streams and Regional Weather Influenced by Modes of Climate}}
\end{flushleft}
\end{par}


\begin{par}
\begin{flushleft}
\texttt{\textbf{Load data}}
\end{flushleft}
\end{par}

\begin{matlabcode}
clear;clc;close
\end{matlabcode}

\begin{par}
\begin{flushleft}
\texttt{Coastline overlay}
\end{flushleft}
\end{par}

\begin{matlabcode}
load('coast_hi_res.mat')
\end{matlabcode}

\begin{par}
\begin{flushleft}
\texttt{AO data}
\end{flushleft}
\end{par}

\begin{matlabcode}
ao = readtable('monthly_AO.txt'); 
\end{matlabcode}

\begin{par}
\begin{flushleft}
\texttt{ENSO data and clean up}
\end{flushleft}
\end{par}

\begin{matlabcode}
nino = importdata('nina34.data.txt',' ',1);
nino = nino.data;
nino(1:2,:) = [];
nino(72,:) = [];
ninodata = nino(:,2:13)';
ninodata = ninodata(:); % concat columns into 1-d array
\end{matlabcode}

\begin{par}
\begin{flushleft}
\texttt{250hpa surface wspd data and clean up}
\end{flushleft}
\end{par}

\begin{matlabcode}
tmp = ncread('data1.nc','u');
wspd = squeeze(tmp(:,:,9,:)); % index 9 is 250hpa
lon = ncread('data1.nc','X');
lat = ncread('data1.nc','Y');
mnths = double(ncread('data1.nc','T')); % months since 1960-01-01 
% want to get months since 1950 Jan 1 instead
% get time array of months since Jan 1950 in datenum
for i=1:length(mnths)
time(i) = addtodate(datenum(1960,1,1,12,0,0),floor(mnths(i)),'month'); 
end
time = time';
\end{matlabcode}

\begin{par}
\begin{flushleft}
\texttt{Find largest common time array between AO, ENSO and WSPD.}
\end{flushleft}
\end{par}

\begin{par}
\begin{flushleft}
\texttt{It is the AO index which spans Jan 1950 to Oct 2020 (850x1 months)}
\end{flushleft}
\end{par}

\begin{matlabcode}
tnino = datenum(1950,(1:852)',1);
knino = find(tnino>=tnino(1) & tnino<=time(end)); 
kwind = find(time>=tnino(1) & time<=time(end));
wspd = wspd(:,:,kwind); 
time = time(kwind);
ninodata = ninodata(knino);
ao = ao.Var3;
clearvars nino kwind tnino knino mnths tmp p i
\end{matlabcode}


\begin{par}
\begin{flushleft}
\texttt{\textbf{Take out mean and seasonal cycle from wspd at each grid point to get wspd anomaly}}\texttt{.}
\end{flushleft}
\end{par}

\begin{matlabcode}
arg = 2*pi*time/12;
xx = [ones(numel(time),1) cos(arg) sin(arg)];
bb = NaN(3,numel(lon),numel(lat));
for i=1:numel(lon)
    for j=1:numel(lat)
        bb(:,i,j) = regress(squeeze(wspd(i,j,:)),xx);
    end
end
wspd_anom = NaN(numel(lon),numel(lat),numel(time));
for i=1:numel(lon)
    for j=1:numel(lat)
    tmp = xx*squeeze(bb(:,i,j));
%   850x1 = 850x3 * 3x1 
    wspd_anom(i,j,:) = squeeze(wspd(i,j,:))-tmp;
    end
end
\end{matlabcode}


\begin{par}
\begin{flushleft}
\texttt{\textbf{Regress wind speed anomalies with AO, ENSO time series}}
\end{flushleft}
\end{par}

\begin{matlabcode}
b = NaN(numel(lon),numel(lat),3);
x = [ninodata ao ones(850,1)];
for i=1:numel(lat)
    for j=1:numel(lon)
        tmp = squeeze(double(wspd_anom(j,i,:)));
        b(j,i,:) = regress(tmp(:),x);
    end
end
\end{matlabcode}

\begin{par}
\begin{flushleft}
\texttt{Also get }\texttt{\textbf{correlation coefficients}}\texttt{, to see how good wind speed and the respective time series correlate!}
\end{flushleft}
\end{par}

\begin{matlabcode}
c1 = NaN(144,73); % ENSO correlation coeff
c2 = NaN(144,73); % AO correlation coeff
for i=1:numel(lat)
    for j=1:numel(lon)
        tmp = corrcoef(wspd_anom(j,i,:),ninodata);
        c1(j,i) = tmp(1,2);
        tmp = corrcoef(wspd_anom(j,i,:),ao);
        c2(j,i) = tmp(1,2);
    end
end
\end{matlabcode}


\begin{par}
\begin{flushleft}
\texttt{Rearrange }\texttt{\textbf{coast contours}}\texttt{ }\texttt{\textbf{data,}}\texttt{ }\texttt{\textbf{wspd-AO regression coefficients, }}\texttt{and }\texttt{\textbf{wspd-AO correlation coefficients }}\texttt{to center Arctic Oscillation plot}
\end{flushleft}
\end{par}

\begin{matlabcode}
% coast contour data 
xc2 = xc; yc2 = yc;
k = find(xc<=180);
xc2(k) = xc2(k)+180;
kk = find(xc>180);
xc2(kk) = xc2(kk)-180;
% AO-wspd regression data 
m1 = find(lon<=180);
m2 = find(lon>180);
b2 = [b(m2,:,2);b(m1,:,2)];
% AO-wspd correlation data
c2 = [c2(m2:end,:);c2(1:m2-1,:)];
clearvars i j k kk m1 m2 tmp 
\end{matlabcode}


\begin{par}
\begin{flushleft}
\texttt{\textbf{Plot ENSO maps (regression + correlation)}}
\end{flushleft}
\end{par}

\begin{matlabcode}
figure(1)
clf
subplot(2,1,1)
pcolor(lon,lat,squeeze(b(:,:,1))');hold on
shading interp
cc = colorbar;
ylabel(cc,'Wspd-Modal multiplier')
caxis([-3 3])
title('Regression ENSO')
xlabel('Longitude (E)')
ylabel('Latitude (N)')
xticks([0:57.5:357.5])
xticklabels([0 60 120 180 -120 -60 0])
scatter(xc,yc,0.01,'.','k')
subplot(2,1,2)
pcolor(lon,lat,c1');hold on
shading interp
cc = colorbar;
ylabel(cc,'Correlation')
title('Correlation ENSO')
xlabel('Longitude (E)')
ylabel('Latitude (N)')
xticks([0:57.5:357.5])
xticklabels([0 60 120 180 -120 -60 0])
scatter(xc,yc,0.01,'.','k')
\end{matlabcode}
\begin{center}
\includegraphics[width=\maxwidth{72.25288509784245em}]{figure_0}
\end{center}
\begin{matlabcode}
% set(gcf, 'Units', 'Normalized', 'OuterPosition', [0.15, 0.3, 0.5, 0.7]);
% saveas(gcf,'/Users/tyler/Desktop/FA20/SIO175/final_proj/figs/ENSO.png')
\end{matlabcode}


\begin{par}
\begin{flushleft}
\texttt{\textbf{Plot AO maps (regression + correlation)}}
\end{flushleft}
\end{par}

\begin{matlabcode}
figure(2)
clf
subplot(2,1,1)
pcolor(lon,lat,b2');hold on
shading interp
cc = colorbar;
ylabel(cc,'Wspd-Modal multiplier')
caxis([-3 3])
title('Regression AO')
xlabel('Longitude (E)')
ylabel('Latitude (N)')
xticks([lon(1):57.5:lon(end)])
xticklabels([-180 -120 -60 0 60 120 180])
scatter(xc2,yc2,0.01,'.','k')
subplot(2,1,2)
pcolor(lon,lat,c2');hold on
shading interp
cc = colorbar;
ylabel(cc,'Correlation')
title('Correlation AO')
xlabel('Longitude (E)')
ylabel('Latitude (N)')
xticks([lon(1):57.5:lon(end)])
xticklabels([-180 -120 -60 0 60 120 180])
scatter(xc2,yc2,0.01,'.','k')
\end{matlabcode}
\begin{center}
\includegraphics[width=\maxwidth{72.25288509784245em}]{figure_1}
\end{center}
\begin{matlabcode}
set(gcf, 'Units', 'Normalized', 'OuterPosition', [0.15, 0.3, 0.5, 0.7]);
\end{matlabcode}


\begin{par}
\begin{flushleft}
\texttt{\textbf{Plot mean wind speed map during El Nino and La Nina}}
\end{flushleft}
\end{par}

\begin{matlabcode}
figure(3)
clf
% load in another data set to easily find el nino events vs la nina events
oni = readtable('oni.ascii.txt');
oni = oni.ANOM;
pos = find(oni>0);
neg = find(oni<0);
pwspd = mean(wspd_anom(:,:,pos),3); 
nwspd = mean(wspd_anom(:,:,neg),3);
subplot(2,1,1)
pcolor(lon,lat,pwspd');hold on
shading interp
title('Mean 250hPa Wind speed during El Nino')
c = colorbar;
ylabel(c,'(m/s)')
xticks([0:57.5:357.5])
xticklabels([0 60 120 180 -120 -60 0])
scatter(xc,yc,1,'k')
subplot(2,1,2)
pcolor(lon,lat,nwspd');hold on
shading interp
title('Mean 250hPa Wind speed during La Nina')
c = colorbar;
ylabel(c,'(m/s)')
scatter(xc,yc,1,'k')
\end{matlabcode}
\begin{center}
\includegraphics[width=\maxwidth{72.25288509784245em}]{figure_2}
\end{center}
\begin{matlabcode}
% set(gcf, 'Units', 'Normalized', 'OuterPosition', [0.15, 0.3, 0.5, 0.7]);
clearvars c cc i j neg pos nwspd pwspd oni tmp 
\end{matlabcode}


\begin{par}
\begin{flushleft}
\texttt{\textbf{Test single linear regression vs multiple}}
\end{flushleft}
\end{par}

\begin{par}
\begin{flushleft}
\texttt{\textbf{Plots look the same}}
\end{flushleft}
\end{par}

\begin{matlabcode}
% figure(4);clf
% btmp = NaN(numel(lon),numel(lat),2);
% x = [ninodata ones(850,1)];
% for i=1:numel(lat)
%     for j=1:numel(lon)
%         tmp = squeeze(double(wspd_anom(j,i,:)));
%         btmp(j,i,:) = regress(tmp(:),x);
%     end
% end
% pcolor(lon,lat,squeeze(btmp(:,:,1))')
% shading interp
% colorbar
% hold on
% scatter(xc,yc,0.01,'k')
\end{matlabcode}

\end{document}
